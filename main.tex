\input{myarticlepreamble}
\input{yliow}
\renewcommand\TITLE{File I/O}
\begin{document}
\topmatter

\section{Options}

There are many ways to perform C/C++ file I/O:
\begin{itemize}
  \li Using C++ fstreams
  \li Using C file I/O -- fopen, fread, fwrite, etc.
  \li Using POSIX standard -- open, read, write, etc.
  \li mmap
\end{itemize}


What happens when file pointer is moved beyond eof?

What are all the possible errors?

What happens when you open a file for r+ and the file does not exists?
\newpage
\myinput{cfileio.tex}



\newpage
\myinput{posix.tex}

WARNING: The problem with POSIX standard is that not all OS are POSIX standard
compliant.
So if you use POSIX, you code will probably not be
portable.
But if you don't care about portability (and want to write for UNIX/Linux)
and speed is important, then POSIX is the way to go.



\end{document}
